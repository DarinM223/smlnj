%!TEX root = manual.tex
%
\chapter{Code Interface}
\label{chap:code-interface}

In this section, we describe the default translation of \asdl{} definitions to target
languages and describe some of the runtime assumptions that users need to be aware of
when using the generated code.

\section{Translation to \sml{}}

The translation from an \asdl{} specification to \sml{} code is straightforward.
\asdl{} modules map to \sml{} structures, \asdl{} product types map to either tuples or records,
and \asdl{} sum types map the \sml{} datatypes.
\tblref{tbl:asdl-to-sml} summarizes this translation.
If an \asdl{} identifier conflicts with an \sml{} keyword or pervasive identifier, then the translation
adds a trailing prime character (\lstinline[language=SML]!'!) to the identifier.

For an \asdl{} module \lstinline[mathescape=true]@$M$@, we generate an \sml{} signature
and several \sml{} structures:
\begin{code}\begin{lstlisting}[language=SML,mathescape=true]
structure $M$ = struct ... end
signature $M$_PICKLE = sig ... end
structure $M$MemoryPickle : $M$_PICKLE = struct ... end
structure $M$FilePickle : $M$_PICKLE = struct ... end
structure $M$SExpPickle : $M$_PICKLE = struct ... end (* optional *)
\end{lstlisting}\end{code}%
where \lstinline[language=SML,mathescape=true]@$M$@ structure contains the type definitions
for the \asdl{} specification,
\lstinline[language=SML,mathescape=true]@$M$MemoryPickle@ structure
implements functions to convert between the types and byte vectors, and the
\lstinline[language=SML,mathescape=true]@$M$FilePickle@ structure implements functions to
read and write pickles from binary files.
The optional \lstinline[language=SML,mathescape=true]@$M$SExpPickle@ structure implements
functions to  read and write textual pickles in S-Expression syntax.\footnote{
  Currently, only output of S-Expression pickles is implemented.
}
This module is generated when the ``\texttt{--sexp}'' option is specified
(see see \secref{sec:s-expressions} for more details).

For an \asdl{} source file \texttt{f.asdl}, \asdlgen{} will produce four \sml{} source files.
\begin{description}
  \item[\normalfont\texttt{f.sml}] \mbox{}\\
    contains type definition structures (\eg{},
    \lstinline[mathescape=true]@structure $M$@)
  \item[\normalfont\texttt{f-pickle.sig}] \mbox{}\\
    contains memory-pickler signatures (\eg{},
    \lstinline[mathescape=true]@signature $M$_PICKLE@)
  \item[\normalfont\texttt{f-memory-pickle.sml}] \mbox{}\\
    contains memory-pickler structures (\eg{},
    \lstinline[mathescape=true]@structure $M$MemoryPickle@)
  \item[\normalfont\texttt{f-file-pickle.sml}] \mbox{}\\
    contains file-pickler structures (\eg{},
    \lstinline[mathescape=true]@structure $M$FilePickle@)
  \item[\normalfont\texttt{f-sexp-pickle.sml}] \mbox{}\\
    contains the optional S-Expression pickler structures (\eg{},
    \lstinline[mathescape=true]@structure $M$SExpPickle@).
    This file is only generated when the ``\texttt{--sexp}''
    command-line option is specified.
\end{description}%

\begin{table}[tp]
  \caption{Translation of \asdl{} types to \sml{}}
  \label{tbl:asdl-to-sml}
  \begin{center}
    \begin{tabular}{|l|p{3in}|}
      \hline
      \textbf{\asdl{} type} & \textbf{\sml{} type} \\
      \hline
      \textit{Named types ($T$)} &  ($\widehat{T}$) \\[0.25em]
      \lstinline!bool! & \lstinline[language=SML]!bool! \\[0.5em]
      \lstinline!int! & \lstinline[language=SML]!int! \\[0.5em]
      \lstinline!uint! & \lstinline[language=SML]!word! \\[0.5em]
      \lstinline!integer! & \lstinline[language=SML]!IntInf.int! \\[0.5em]
      \lstinline!string! & \lstinline[language=SML]!string! \\[0.5em]
      \lstinline!identifier! & \lstinline[language=SML]!Atom.atom! \\[0.5em]
      \lstinline[language=ASDL,mathescape=true]@$t$@ & \lstinline[language=SML,mathescape=true]!$t$! \\[0.5em]
      \lstinline[language=ASDL,mathescape=true]@$M$.$t$@ & \lstinline[language=SML,mathescape=true]!$M$.$t$! \\[0.5em]
      \hline
      \textit{Type expressions ($\tau$)} &  ($\widehat{\tau}$) \\[0.25em]
      \lstinline[language=ASDL,mathescape=true]@$T$@ & \lstinline[language=SML,mathescape=true]!$\widehat{T}$! \\[0.5em]
      \lstinline[language=ASDL,mathescape=true]@$T$?@ & \lstinline[language=SML,mathescape=true]!$\widehat{T}$ option! \\[0.5em]
      \lstinline[language=ASDL,mathescape=true]@$T$*@ & \lstinline[language=SML,mathescape=true]!$\widehat{T}$ list! \\[0.5em]
      \hline
      \textit{Product types ($\rho$)} & ($\widehat{\rho}$) \\[0.25em]
      \lstinline[language=ASDL,mathescape=true]@($\tau_1$, $\ldots$, $\tau_n$)@
        & \lstinline[language=SML,mathescape=true]!$\widehat{\tau}_1$ * $\cdots$ * $\widehat{\tau}_n$! \\[0.5em]
      \lstinline[language=ASDL,mathescape=true]@($\tau_1$ $f_1$, $\ldots$, $\tau_n$ $f_n$)@
        & \lstinline[language=SML,mathescape=true]!{$f_1$ : $\widehat{\tau}_1$, $\ldots$, $f_n$ : $\widehat{\tau}_n$}! \\[0.5em]
      \hline
      \textit{Type definitions} & \\[0.25em]
      \lstinline[language=ASDL,mathescape=true]@$t$ = $\rho$@
        & \lstinline[language=SML,mathescape=true]!type $t$ = $\widehat{\rho}$! \\[0.5em]
      \lstinline[language=ASDL,mathescape=true]@$t$ = $C_1$ | $\cdots$ | $C_n$@
        & \lstinline[language=SML,mathescape=true]!datatype $t$ = $C_1$ | $\cdots$ | $C_n$! \\[0.5em]
      \lstinline[language=ASDL,mathescape=true]@$t$ = $C_1$($\rho_1$) | $\cdots$ | $C_n$($\rho_n$)@
        & \lstinline[language=SML,mathescape=true]!datatype $t$ = $C_1$ of $\widehat{\rho}_1$ | $\cdots$ | $C_n$ of $\widehat{\rho}_n$! \\[0.25em]
      \hline
    \end{tabular}%
  \end{center}%
\end{table}%

\subsection{Sharing}
Shared types change the representation of a pickle from a tree to a DAG
(Directed Acyclic Graph).
If an ASDL specification includes shared types (\ie{}, uses of the ``\kw{!}''
type operator, then a more complicated interface to the pickler/unpickler
is required.
Basically, we need to augment the input and output streams with sharing
state that tracks repeated occurrences of the same value.
For unpickling, \asdlgen{} generates additional data structures to track
shared values (see \secref{sec:shared-pickle-rep} for details), which changes
the unpickling interface but does not require any user-supplied code.

For pickling, however, we require that the user implement a mapping for
each shared type in the specification.
We expect that for typical use cases, shared values will have a unique key
embedded in their representation that can be used to efficiently detect sharing on
output.
Since \asdlgen{} cannot easily determine what the key value might be, we
require user-defined code.
For each shared type $\tau$, the user must implement a function with the
type ``\lstinline[language=sml,mathescape=true]@($\tau$, 'a) share_map@,''
where
\begin{code}
\begin{lstlisting}[language=sml]
type ('key, 'a) share_map = unit -> {
       insert : 'key * 'a -> unit,
       find : 'key -> 'a option
     }
\end{lstlisting}%
\end{code}%


For example, say that we had the following \asdl{} specification fragment
as part of a compiler's intermediate representation:
\begin{code}
\begin{lstlisting}[language=asdl]
  var_rep = (int id, string name, ty ty)
  var = var_rep!
\end{lstlisting}%
\end{code}%
where the ``\cd{id}'' field is a unique ID for each variable.
Then we could implement the sharing map for variables using the
``\cd{IntHashTable}'' module from the \smlnj{} Library:
\begin{code}
\begin{lstlisting}[language=sml]
val var_sharemap = fn () => let
      val tbl = IntHashTable.mkTable (32, Fail "var_sharemap")
      in {
        insert = fn (x : var, y) => IntHashTable.insert tbl (#id x, y),
        find = fn (x : var) => IntHashTable.find tbl (#id x)
      } end
\end{lstlisting}%
\end{code}%

\subsection{CM support}
The SML/NJ Compilation Manager (CM) knows about \asdl{} files (as of version 110.84).
If one specifies ``\texttt{foo.asdl}'' in the file list of a \texttt{.cm} file, CM will
infer the generation of the five \sml{} files as described above.

\section{Translation to \Cplusplus{}}

The translation of an \asdl{} specification to \Cplusplus{} is more complicated than for \sml{}.
For each \asdl{} module, we define a corresponding \Cplusplus{} namespace.

\subsection{Translation of named types}

The following table summarizes how type names are mapped to \Cplusplus{} type expressions.
%
\begin{center}
  \begin{tabular}{|p{2in}|p{3in}|}
    \hline
      \textbf{Named \asdl{} type ($T$)} &  \textbf{\Cplusplus{} type $\widehat{T}$} \\
    \hline
      \lstinline!bool! & \lstinline[language=c++]!bool! \\[0.5em]
      \lstinline!int! & \lstinline[language=c++]!int! \\[0.5em]
      \lstinline!uint! & \lstinline[language=c++]!unsigned int! \\[0.5em]
      \lstinline!integer! & \lstinline[language=c++]!asdl::integer! \\[0.5em]
      \lstinline!string! & \lstinline[language=c++]!std::string! \\[0.5em]
      \lstinline!identifier! & \lstinline[language=c++]!asdl::identifier! \\[0.5em]
      \lstinline[language=ASDL,mathescape=true]@$t$@ &
        $\left\{
        \begin{array}{ll}
          t & \text{if $t$ is an \lstinline[language=c++]!enum! type} \\
          \text{\lstinline[language=c++,mathescape=true]@$t$ *@} & otherwise \\
        \end{array}
        \right.$ \\[0.5em]
      \lstinline[language=ASDL,mathescape=true]@$M$.$t$@ & \lstinline[language=c++,mathescape=true]@$M$::$t$@ \\[0.5em]
    \hline
  \end{tabular}%
\end{center}%
%
In the subsequent discussion, we write $\widehat{T}$ to denote the \Cplusplus{} type
that corresponds the the \asdl{} named type $T$.

\subsection{Translation of type expressions}

The translation of \asdl{} type expressions formed by applying a type operator to
a named type $T$ is described in the following table.
%
\begin{center}
  \begin{tabular}{|p{2in}|p{3in}|}
    \hline
      \textbf{\asdl{} type expression ($\tau$)} &  \textbf{\Cplusplus{} type $\widehat{\tau}$} \\[0.25em]
    \hline
      \lstinline[language=ASDL,mathescape=true]@$T$@ & \lstinline[language=c++,mathescape=true]@$\widehat{T}$@ \\[0.5em]
      \lstinline[language=ASDL,mathescape=true]@$T$?@ &
        $\left\{
        \begin{array}{ll}
          \text{\lstinline[language=c++,mathescape=true]@asdl::option< $\widehat{T}$ >@}
            & \text{if $t$ is an \lstinline[language=c++]!enum! type} \\
          \text{\lstinline[language=c++,mathescape=true]@$\widehat{T}$@} & otherwise \\[0.5em]
        \end{array}
        \right.$ \\[0.5em]
      \lstinline[language=ASDL,mathescape=true]@$T$*@ & \lstinline[language=c++,mathescape=true]@std::vector< $\widehat{T}$ >@ \\[0.5em]
    \hline
  \end{tabular}%
\end{center}%
%
Notice that the for option types, we use \lstinline[language=c++]@nullptr@ to
represent the empty option for those \asdl{} types that are represented by
\Cplusplus{} pointer types.
For other \asdl{} types, which are not represented by pointers, we wrap the type
with the template class ``\lstinline[language=c++]@option<>@''
from the \asdl{} library.
This class provides methods for testing if an option is empty (\lstinline[language=c++]@isEmpty@
and for getting the value when it is not (\lstinline[language=c++]@valOf@).

\subsection{Translation of type definitions}

The translation of type definitions depends on the form of the right-hand-side
of the definition.
For product types, we map the fields of the product to a sequence of \Cplusplus{}
member declarations (as described below) enclosed in a \Cplusplus{} \kw{struct} type.
For enumerations, we use \Cplusplus{} scoped enumerations to represent the type.
For other sum types, we define an abstract base class for the type with subclasses
for each constructor.
This translation is summarized in the following table, where we are using
\lstinline[mathescape=true]@$\rho$@ to represent the fields of an \asdl{}
product type and \lstinline[mathescape=true]@$\widehat{\rho}$@ for its translation:
%
\begin{center}
  \begin{tabular}{|p{2in}|p{3in}|}
    \hline
      \textbf{\asdl{} type definition} &  \textbf{\Cplusplus{} type} \\
    \hline
      \lstinline[language=ASDL,mathescape=true]@$t$ = $\rho$@
        & \lstinline[language=c++,mathescape=true]@struct $t$ { $\hat{\rho}$ };@ \\[0.5em]
      \lstinline[language=ASDL,mathescape=true]@$t$ = $C_1$ | $\cdots$ | $C_n$@
        & \lstinline[language=c++,mathescape=true]@class enum $t$ { $C_1$, $\ldots$, $C_n$ };@ \\[0.5em]
      \lstinline[language=ASDL,mathescape=true]@$t$ = $C_1$($\rho_1$) | $\cdots$ | $C_n$($\rho_n$)@
        &
\vspace*{-1em}
\begin{lstlisting}[language=c++,mathescape=true]
class $t$ { $\cdots$ };
class $C_1$ : public $t$ {
  private: $\hat{\rho}_1$
  $\cdots$
};
$\cdots$
class $C_n$ : public $t$ {
  private: $\hat{\rho}_n$
  $\cdots$
};
\end{lstlisting}%
      \\[0.25em]
    \hline
  \end{tabular}%
\end{center}%

The translation of product types is described by the following table:
%
\begin{center}
  \begin{tabular}{|p{2in}|p{3in}|}
    \hline
      \textbf{\asdl{} product type ($\rho$)} &  \textbf{\Cplusplus{} type $\widehat{\rho}$} \\
    \hline
      \lstinline[language=ASDL,mathescape=true]@($\tau_1$, $\ldots$, $\tau_n$)@
        & \lstinline[language=c++,mathescape=true]@$\hat{\tau}_1$ _v1; $\ldots$ $\hat{\tau}_n$ _vn@ \\[0.5em]
      \lstinline[language=ASDL,mathescape=true]@($\tau_1$ $f_1$, $\ldots$, $\tau_n$ $f_n$)@
        & \lstinline[language=c++,mathescape=true]@$\hat{\tau}_1$ _$f_1$; $\ldots$ $\hat{\tau}_n$ _$f_n$@ \\[0.5em]
    \hline
  \end{tabular}%
\end{center}%

\subsection{Pickling and unpickling operations}

The code generator for the \Cplusplus{} view uses a mix of methods, static methods,
and overloaded functions to implement the pickler and unpickler operations.

\subsection{Memory management}

\subsection{Makefile support}
Currently there is no support for generating makefile dependencies or rules,
but it may be added in a future release.


\section{The Rosetta Stone for Sum Types}
\label{sec:rosetta-stone}

For languages that support algebraic data types, \asdlgen{} maps sum types directly
to the language's mechanism (\eg{}, \lstinline[language=SML]!datatype! declarations
in \sml{}).
For class-based object-oriented languages, like \Cplusplus{}, \asdlgen{} maps
sum types to abstract base classes and the constructors to individual subclasses.
The previous example written in \sml{} would be
\begin{quote}\begin{lstlisting}[language=SML]
structure M =
  struct
    datatype sexpr
      = Int of (int)
      | String of (string)
      | Symbol of (identifier)
      | Cons of (sexpr * sexpr)
      | Nil
  end
\end{lstlisting}\end{quote}%
and in \Cplusplus{} it translates to
\begin{quote}\begin{lstlisting}[language=c++]
namespace M {

    struct sexpr {
        enum tag {
            _Int, _String, _Symbol, _Cons, _Nil
        };
        tag _tag;
        sexpr (tag t) : _tag(t) { }
        virtual ~sexpr ();
    };

    struct Int : public sexpr {
        int _v1;
        Int (int v) : sexpr(sexpr::_Int), _v1(v) { }
        ~Int () { }
    };

    struct String : public sexpr {
        std::string _v1;
        String (const char *v) : sexpr(sexpr::_String), _v1(v) { }
        String (std::string const &v) : sexpr(sexpr::_String), _v1(v) { }
        ~String () { }
    };

    struct Symbol : public sexpr { ... };

    struct Cons : public sexpr { ... };

    struct Nil : public sexpr { ... };

}
\end{lstlisting}\end{quote}%
