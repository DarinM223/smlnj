%!TEX root = manual.tex
%
\chapter{Pickles}
\label{chap:pickles}

One of the most important features of \asdlgen{} is that it automatically
produces functions that can read and write the data structures it generates
to and from a platform and language independent external
representation.
This process of converting data structures in memory into a
sequence of bytes on the disk is referred to as \emph{pickling}.\footnote{
  It is also called \emph{serialization}.
}
Since it is possible to generate data structures and pickling code for
any of the supported languages from a single \asdl{} specification,
\asdlgen{} provides an easy
and efficient way to share complex data structures among these languages.

The \asdl{} pickle format requires that both the reader and writer
of the pickler agree on the type of the pickle.
Other than constructor tags for sum types, there is no explicit type
information in the pickle.
In the case of an error, the behavior is undefined.
It is also important that when pickling/unpickling to/from files, that the
files be opened in binary mode to prevent line feed translations from corrupting
the pickle.

\section{Binary Pickle Format}

Since \asdl{} data structures have a tree-like form, they can be represented
linearly with a simple prefix encoding.
This encoding is used by both the memory and file picklers and
is described in this section.

\subsection{Primitive types}

\begin{description}
  \item[bool] \mbox{}\\
    Boolean values are represented by \lstinline!1! (false) or \lstinline!2! (true)
    and are encoded in one byte.  We reserve the value zero to represent the
    empty value for the type \lstinline@bool?@.

  \item[int] \mbox{}\\
    The \lstinline!int! type provides 30-bits of signed precision encoded in one to
    four bytes.
    The top two bits of the first byte (bit~6--7) specify the number of additional
    bytes in the encoding and bit~5 specifies the sign of the number.
    Thus values in the range -32 to 31 can be encoded in one byte,
    -8192 to 8191 in two bytes, \etc{}
    A negative number $n$ is represented as the positive number $-(n+1)$.

  \item[uint] \mbox{}\\
    The \lstinline!uint! type provides 30-bits of unsigned precision encoded in one
    to four bytes.
    As with the \lstinline!int! type, the top two bits of the first byte specify the
    number of additional bytes in the encoding.
    Thus values in the range 0 to 63 can be encoded in one byte,
    0 to 16383 unsigned in two bytes, \etc{}

  \item[integer] \mbox{}\\
    The \asdl{} \lstinline!integer! type is represented with a variable-length,
    big-endian, signed-magnitude encoding.
    The high bit of each byte indicates if the byte is the last byte of
    the encoding.
    The bit~6 of the most significant byte is used to determine the
    sign of the value.
    Thus, numbers in the range of -63 to 63 are encoded in one byte.
    Numbers outside of this range require an extra byte for every seven bits
    of precision required.

  \item[string] \mbox{}\\
    Strings are represented with a length-header that describe how many more
    8-bit bytes follow for the string and then the data for the string in bytes.
    The length-header is encoded as a \lstinline!uint! value, thus strings are limited
    to 1,073,741,823 characters.

  \item[identifier] \mbox{}\\
    Identifiers are represented as if they were strings.

  \item[tags] \mbox{}\\
    Tags are an internal type that \asdl{} uses to represent enumerations and
    tagged sums.
    If the number of distinct tags required for a type is less than 256, then
    the tag is represented as a single byte.
    Otherwise, the tag value is encoded as a \lstinline!uint! value.
\end{description}%

\subsection{Product types}
The fields of a product type are encoded sequentially (left to right)
without any initial tag.

\subsection{Enumeration types}
Enumeration types are represented by a set of tag values, one per constructor
on the type.
Tag values are assigned in order of constructor definition starting from one.
As with the \lstinline@bool@ type, the value zero is used to encode empty option values.

If the enumeration type has only a single constructor, then it is implicit in
the encoding (\ie{}, no space is used to represent it).

\subsection{Sum types}
Non-enumeration sum types begin with a unique tag to identify the constructor
followed by the fields of the constructor.
Tag values are assigned in order of constructor definition starting from one (the
value zero is used to encode empty option values).
As with product types, fields are packed left to right based on the order
in the definition.
If there are any attribute values associated with the type, they are packed left to right
after the tag but before other constructor fields.
  
If the sum type has a single constructor with fields, then it is
encoded without a tag (\ie{}, like a product type).

\subsection{Sequence types}
Sequence types are represented with an integer length-header followed by
that many values of that type.
The length-header is encoded as a \lstinline!uint! value, thus sequences are limited
to at most 1,073,741,823 items.

\subsection{Option types}
The encoding of optional values depends on the base type.
For sum types with more than one constructor (including \lstinline@bool@),
the special tag value of zero is used to denote an empty value and
non-zero values are interpreted as the constructor's tag.
For any other base type, there is an initial byte that is either one or zero.
A zero indicates that the value is empty and no more data follows.
A one indicates that the next value is the value of the optional value.

\subsection{Shared types}
\label{sec:shared-pickle-rep}

Shared types change the representation of a pickle from a tree to a DAG
(Directed Acyclic Graph).  A shared value is the pickle is
either a \emph{definition} occurrence (\ie{}, the first occurrence of that
value), or a \emph{back reference} (\ie{}, subsequent occurrences of the value).
A shared value is represented by an integer followed by the encoding of the
value itself.  For a definitional occurrence, the integer will be zero, and
for subsequent occurrences, the integer will count back to the shared definition
(\eg{}, one maps to the most recent definition, two to the one before it, \etc{}).
For unpickling, this representation only requires that the unpickler maintain
a stack per shared type.  Definitions get pushed onto the stack and back references
extract existing values from the stack.

Note that shared types are currently the only feature of \asdl{} that are
\emph{stateful}; \ie{}, that pickling/unpickling a value depends on previous
values in the stream.

\subsection{Alias types}
Alias types use the encoding of their definition.

\subsection{User-defined primitive types}
User-defined primitive types are pickled/unpickled by user-provided functions (see
\secref{sec:primitive-syntax}).

\section{S-expression Format}
\label{sec:s-expressions}
It is also possible to generate a text-based representation of pickles in S-Expression
syntax.
Primitive values are represented as literals, enumerations are represented
as quoted symbols, and structured values are represented with a parenthesized
expression of the form
\begin{code}\begin{lstlisting}[mathescape=true]
( $\mathit{op}$ $v'_1$ $\cdots$ $v'_n$ )
\end{lstlisting}\end{code}%
where $\mathit{op}$ is an identifier that defines the structure of the value
and the $v'_1,\,\ldots,\,v'_n$ are the encodings of the sub-values.

\subsection{Primitive types}

\begin{description}
  \item[bool] \mbox{}\\
    Boolean values are mapped to the literals \lstinline!#t! and \lstinline!#f!.
    
  \item[numbers] \mbox{}\\
    Values of the ASDL numeric types (\lstinline!int!, \lstinline!uint!, and
    \lstinline!integer!) are represented by decimal literals.

  \item[string and identifier] \mbox{}\\
    These values are represented by string literals.
\end{description}%

\subsection{Product types}
A product value \lstinline[mathescape=true]@($f_1$, $\ldots$, $f_n$)@
is encoded as the
S-expression \lstinline[mathescape=true]@($n$-tuple $f'_1$ $\cdots$ $f'_n$)@,
where the $f'_i$ are the encoded fields of the value.
Unlabeled fields are directly represented by their value, whereas a
field with label $l$ and value $v$ is encoded as
\lstinline[mathescape=true]@($l$ $v'$)@, where $v'$ is the encoding of the field's
value.

\subsection{Sum types}
Nullary constructors are mapped to quoted symbols, while non-nullary constructors are mapped
to an S-expression with the constructor name as the operator and the 

\subsection{Sequence types}
A sequence of values $v_1,\,\ldots,\,v_n$ is encoded as
\lstinline[mathescape=true]@(* $v'_1$ $\cdots$ $v'_n$)@,
where the $v'_i$ are the encoded fields of the sequence.

\subsection{Option types}
An empty option value is represented as \lstinline@(?)@, while a non-empty
option value with contents $v$ is represented as \lstinline[mathescape=true]@(? $v'$)@,
where $v'$ is the encoding of the contents.

\subsection{Alias types}
Alias types use the encoding of their definition.

\subsection{User-defined primitive types}
User-defined primitive types are not yet supported in S-Expression form.

%\section{XML Pickle Format}
% TODO
