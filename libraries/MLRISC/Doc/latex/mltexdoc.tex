\documentclass{article} 
\usepackage{mltex}
\usepackage{wrapfig}
\usepackage{float}
%\usepackage{floatfig}
\usepackage{fancyheadings}
%\usepackage{draftcopy}
%\usepackage{bookman}
\usepackage{utopia}
%\usepackage{times}
%\usepackage{ncntrsbk}
%\usepackage{palatino}

   \setlength{\textwidth}{6.5in}
   \setlength{\evensidemargin}{0in}
   \setlength{\oddsidemargin}{0in}
   \setlength{\textheight}{8in}
   \setlength{\topmargin}{-0.5in}

   \pagestyle{fancyplain}
   %\addtolength{\headwidth}{\marginparsep}
   %\addtolength{\headwidth}{\marginparwidth}

   \newcommand{\edge}[1]{\rightarrow_{#1}}
   \newcommand{\union}{\cup}
   \newcommand{\Union}{\bigcup}
   \newcommand{\overrides}{overrides}
   \newcommand{\defas}{\stackrel{\rm as}{=}}

   \renewcommand{\sectionmark}[1]{\markright{\thesection\ #1}}
   \renewcommand{\subsectionmark}[1]{\markright{\thesubsection\ #1}}
   \newcommand{\Term}[1]{\mbox{\it #1}}
   \lhead[\fancyplain{}{\bfseries\thepage}]%
         {\fancyplain{}{\bfseries\rightmark}}
   \rhead[\fancyplain{}{\bfseries\leftmark}]%
         {\fancyplain{}{\bfseries\thepage}}
   \cfoot{}

   \newenvironment{Figure}{\begin{figure}[htbp]}{\end{figure}}

\begin{document}
   \title{\bf \LARGE \MLTeX} 
   \author{\begin{tabular}{c}
            Allen Leung \\ \\
            New York University \\
            719 Broadway, Rm. 714 \\ 
            New York, NY 10003. \\
            {\tt leunga@cs.nyu.edu}
           \end{tabular}
        \and 
           \begin{tabular}{c}
            Lal George \\ \\
            Bell Laboratories \\
            600--700 Mountain Ave. \\
            Murray Hill, NJ 07974--0636. \\
            {\tt george@research.bell-labs.com}
            \end{tabular}
     }

   \date{\today}
   \bibliographystyle{alpha}

   \maketitle

   \begin{abstract}
 \newdef{\MLTeX} is a special \newdef{\LaTeX} package for writing
\MLRISC{} documentation.  It is similar to the \newdef{latex2html}
tool except that \MLTeX{} has special environments for documenting
Standard ML code.   In addition, there is an accompanying tool
called \newdef{mltex2html} for generating HTML pages.
   \end{abstract}
 

   %
% This is derived from alltt.sty
%
\NeedsTeXFormat{LaTeX2e}
\ProvidesPackage{mltex}[2000/2/12 defines mltex environment]

\usepackage{latexsym}
%\usepackage{psfig}
\usepackage{fancyheadings}
\usepackage{sml}
\usepackage{color}
\usepackage{verbatim}


% Sectioning

\newcommand{\Chapter}[1]{\chapter{#1}}
\newcommand{\Section}[1]{\section{#1}}
\newcommand{\Subsection}[1]{\subsection{#1}}
\newcommand{\Subsubsection}[1]{\subsubsection{#1}}
\newcommand{\Paragraph}[1]{\paragraph{#1}}
\newcommand{\majorsection}[1]{}

% Frames
   \newsavebox{\savepar}
   \newenvironment{boxit}{\begin{lrbox}{\savepar}
   \begin{minipage}[b]{\columnwidth}}
   {\end{minipage}\end{lrbox}\begin{center}\framebox[\columnwidth]{\usebox{\savepar}}\end{center}}
   \newenvironment{Boxit}{\begin{tabular}{|c|}\hline}{\\\hline\end{tabular}}


% Table
\newenvironment{Table}[2]{\begin{tabular}{#1}}{\end{tabular}}

% Images and Figures
\newcommand{\image}[3]{}
%\newcommand{\cpsfig}[1]{\centerline{\psfig{#1}}}

% Formatting
\newenvironment{Bold}{\begingroup\bf}{\endgroup}
\newenvironment{Italics}{\begingroup\it}{\endgroup}
\newenvironment{Emph}{\begingroup\em}{\endgroup}

\newcommand{\italics}[1]{{\it #1}}
\newcommand{\bold}[1]{{\bf #1}}
%\newcommand{\emph}[1]{{\em #1}}

% New definitions
\newcommand{\newdef}[1]{{\em #1}}
\newcommand{\newtype}[1]{{\tt #1}}

% Enviroments
\newenvironment{SML}{\begin{smldisplay}}{\end{smldisplay}}
\newenvironment{methods}{\begin{center}\begin{tabular}{@{\tt}ll}}%
   {\end{tabular}\end{center}}
\newenvironment{address}{}{}

% HTML
\newcommand{\href}[2]{#2\footnote{url: \tt #1}}
\newcommand{\externhref}[2]{#2\footnote{url: \tt #1}}
\newcommand{\mlrischref}[2]{\sml{#2}\footnote{{\bf file:} {\tt #1}}}

\newcommand{\br}[1]{}
\newcommand{\hr}{\rule{1em}{\textwidth}}

% Color
\renewenvironment{color}[1]{}{}

% Misc
\newcommand{\MLRISC}{MLRISC}
\newcommand{\MLTeX}{\mbox{$\mbox{MLT}_{\mbox{E}}\mbox{X}$}}

\endinput

   \bibliography{mlrisc}
\end{document}
